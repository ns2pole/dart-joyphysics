\documentclass[a4paper,11pt]{article}

\usepackage{luatexja}          % ← 必須
\usepackage{luatexja-fontspec} % ← フォント指定用(安全)
\usepackage{amsmath,amssymb}
\usepackage{enumitem}
\usepackage{geometry}
\usepackage{tcolorbox} % プリアンブル

\geometry{margin=15mm}

\begin{document}

\section*{\fbox{問題への取り組み方}}

まず,問題が何を言っているのかを理解する必要があります。  
問題を自分の言葉で語ることができるか,言い換えることができるかを考えてみましょう。
\\
\ 
\\
さて、問題が何を聞いているのか、おおよそ理解した気持ちになれたとして、その次に,問題解決のために何ができるでしょうか。

\section*{\fbox{いかにして問題を解くか}}
\subsection*{[1] 迷子の防止}
\begin{enumerate}[label=1-\arabic*]
  \item 条件をすべて使ったか?
  \item 結論を推測できるか?(証明問題ではない場合)
  \item 結論を示すためには何を示せばよいか?  
  (結論から遡る,結論と同値な命題を作る)
  \item 「詳細」と「俯瞰」を意識する。(計算処理と全体の流れ)
  \item その言葉の定義は何か?\\
\end{enumerate}
\subsection*{[2] 問題解決の糸口}
\begin{enumerate}[label=2-\arabic*]
  \item 具体例を計算し,リストを作成する。  
  次に,作成したリストにパターンを見出せるか?
  \item 似た問題を解いたことはないか?  
  (もしあれば)その問題に帰着させることはできるか?
  \item 単純化(難易度をコントロールする)
  \item 特殊化した問題を解くことはできるか?部分的に問題を解くことはできるか?
  \item 絵,図,グラフを描くことはできるか?
  \item 否定,余事象を考える\\
\end{enumerate}

% \subsection*{[3] 検証・結果の吟味}

% 具体的な数値を代入して検算を行う。  
% 特に極端な例で検算を行う($x=0$,$\displaystyle \lim_{x\to \pm\infty}$)
\newpage

\section*{\fbox{補足}}
\subsection*{[1-1] 条件の整理}
条件が多い場合は,既に使った条件にアンダーラインを引くとよい。

\subsection*{[1-3] 結論から遡る例}
\begin{center}
\begin{tcolorbox}[
  width=0.65\textwidth,
  boxrule=0.8pt,
  arc=6pt,
  colframe=black,
  colback=gray!5,
  left=8pt,right=8pt,top=6pt,bottom=6pt
]
\centering
$a b c = 1,\quad a+b+c = ab+bc+ca$ のとき,\\
$a,b,c$ のうち少なくとも1つは $1$ であることを示せ。
\end{tcolorbox}
\end{center}
\[
\Downarrow \text{結論から遡る}
\]
\begin{center}
\begin{tcolorbox}[
  width=0.8\textwidth,
  boxrule=0.6pt,
  arc=6pt,
  colframe=black,
  colback=white,
  left=8pt,right=8pt,top=8pt,bottom=8pt
]
\[
\begin{aligned}
\cdots \\
\iff\;& abc-(ab+bc+ca)+(a+b+c)-1=0\\
\iff\;& (a-1)(b-1)(c-1)=0 \\
\iff\;& a,b,c \text{ のうち少なくとも1つは } 1 \\
\end{aligned}
\]
\end{tcolorbox}
\end{center}

のように,結論から逆向きに辿ると容易くなる。

\subsection*{[2-1] 具体例を計算し,リストを作成する例}
\begin{center}
\begin{tcolorbox}[
  width=0.95\textwidth,
  boxrule=0.8pt,
  arc=6pt,
  colframe=black,     % ← 枠線の色
  colback=gray!5,     % ← 背景色(薄いグレー)
  left=8pt,right=8pt,top=6pt,bottom=6pt
]
\center
数列 $\{a_n\}$ は漸化式$\displaystyle a_1 = 7,\quad a_{n+1} = \frac{4a_n - 9}{a_n - 2}$を満たすとする。一般項 $\{a_n\}$ を求めよ。
\end{tcolorbox}
\end{center}
\[
\Downarrow \text{具体計算とリストの作成}
\]

\begin{center}
\begin{tcolorbox}[
  width=0.5\textwidth,
  boxrule=0.8pt,
  arc=6pt,
  colframe=black,     % ← 枠線の色
  colback=white,
  left=8pt,right=8pt,top=6pt,bottom=6pt
]
\renewcommand{\arraystretch}{1.8}
\[
\begin{array}{c|cccc}
n
 & 1 & 2 & 3 & 4 \\[-3pt] \hline
a_n
 & 7
 & \dfrac{19}{5}\rule{0pt}{14pt}
 & \dfrac{31}{9}\rule{0pt}{14pt}
 & \dfrac{43}{13}\rule{0pt}{14pt}
\end{array}
\]
\end{tcolorbox}
\end{center}
\begin{center}
\textbf{第1項:\ } $a_1 = 7$,\  \ \textbf{第2項:\ }$\displaystyle a_2= \frac{4\cdot 7 - 9}{7 - 2}= \frac{28 - 9}{5}= \frac{19}{5}$
\\[10pt]
\textbf{第3項:\ }$\displaystyle
a_3
= \frac{4\cdot \frac{19}{5} - 9}{\frac{19}{5} - 2}
= \frac{\frac{76}{5} - \frac{45}{5}}{\frac{19}{5} - \frac{10}{5}}
= \frac{\frac{31}{5}}{\frac{9}{5}}
= \frac{31}{9}
$,\ \  
\textbf{第4項:\ }
$\displaystyle a_4
= \frac{4\cdot \frac{31}{9} - 9}{\frac{31}{9} - 2}
= \frac{\frac{124}{9} - \frac{81}{9}}{\frac{31}{9} - \frac{18}{9}}
= \frac{\frac{43}{9}}{\frac{13}{9}}
= \frac{43}{13} $\\[10pt]
\end{center}
これより$\displaystyle a_1=\frac{7}{1}$と読み替えれば、$\displaystyle a_n=\frac {12n-5}{4n-3}$と推測できる。(あとは帰納法を使えば良い。)
\newpage
\subsection*{[2-3] 単純化の例}
\begin{center}
\begin{tcolorbox}[
  width=0.85\textwidth,
  boxrule=0.8pt,
  arc=6pt,
  colframe=black,
  colback=gray!5,
  left=8pt,right=8pt,top=6pt,bottom=6pt
]
\centering

生徒9人を次の3つのグループに分ける分け方は何通りあるか。\\[6pt]
  (1) 4人,3人,2人の3つのグループに分ける。\\[2pt]
  (2) 3人ずつ,3つのグループ $A,B,C$ に分ける。\\[2pt]
  (3) 3人ずつ,3つのグループに分ける。

\end{tcolorbox}
\end{center}
\[
\Downarrow\ \text{単純化}
\]
\begin{center}
\begin{tcolorbox}[
  width=0.75\textwidth,
  boxrule=0.8pt,
  arc=6pt,
  colframe=black,
  colback=white,
  left=8pt,right=8pt,top=6pt,bottom=6pt
]
\centering

生徒4人を2つのグループに分ける分け方は何通りあるか。\\[6pt]
(1) 3人,1人の2グループに分ける。\\
(2) 2人ずつ,区別のある2グループ $X,Y$ に分ける。\\
(3) 2人ずつ,区別のない2グループに分ける。
\end{tcolorbox}
\end{center}

\paragraph{(1) }
3人のグループを $X$ ,残りの1人を $Y$ とする。
具体的に書き出すと,

\[
\begin{aligned}
X:&\{B,C,D\}\quad Y:\{A\} \\
X:&\{A,C,D\}\quad Y:\{B\} \\
X:&\{A,B,D\}\quad Y:\{C\} \\
X:&\{A,B,C\}\quad Y:\{D\}
\end{aligned}
\]

以上より,分け方は $4$ 通りである。

\paragraph{(2) }
まずグループ $X$ に入る2人を決める。
残りの2人は自動的にグループ $Y$ に入る。

\[
\begin{aligned}
X:&\{A,B\}\quad Y:\{C,D\} \\
X:&\{A,C\}\quad Y:\{B,D\} \\
X:&\{A,D\}\quad Y:\{B,C\} \\
X:&\{B,C\}\quad Y:\{A,D\} \\
X:&\{B,D\}\quad Y:\{A,C\} \\
X:&\{C,D\}\quad Y:\{A,B\}
\end{aligned}
\]

以上より,分け方は $6$ 通りである。

\paragraph{(3) }(2)で書き出した分け方のうち,
$X$ と $Y$ を入れ替えても同じとみなされるものをまとめる。

\[
\begin{aligned}
&\{A,B\}|\{C,D\} \\
&\{A,C\}|\{B,D\} \\
&\{A,D\}|\{B,C\}
\end{aligned}
\]

たとえば(2)の
$
X:\{A,B\},\,Y:\{C,D\}
\quad\text{と}\quad
X:\{C,D\},\,Y:\{A,B\}
$
は(3)では同一の分け方とみなす。以上より,分け方は $3$ 通りである。

\paragraph{\fbox{考察}\\[10pt]}
具体的に単純化を行い、リスト化できる場合の数の問題に難易度を下げることで,グループに名前がある時とない時の区別が分かりやすくなった。そして、\textbf{この考え方を9人の場合にもそのまま適用できる。}
\newpage
\subsection*{[2-4] 部分的に問題を解くことはできるか? の例}
\begin{center}
\begin{tcolorbox}[
  width=0.95\textwidth,
  boxrule=0.8pt,
  arc=6pt,
  colframe=black,
  colback=gray!5,
  left=8pt,right=8pt,top=6pt,bottom=6pt
]
\centering
$2 \le x \le 6$ において常に,$x^2 - 4ax + 4a + 8 > 0$
が成り立つような実数 $a$ の範囲を求めよ。
\end{tcolorbox}
\end{center}
\[
\Downarrow \text{部分問題}
\]
\begin{center}
\begin{tcolorbox}[
  width=0.6\textwidth,
  boxrule=0.6pt,
  arc=6pt,
  colframe=black,
  colback=white,
  left=8pt,right=8pt,top=6pt,bottom=6pt
]
\centering
$a=0$ は答えの領域に含まれるか?
\end{tcolorbox}
\end{center}

定数 \(a\) に具体的な値を代入してみる。ここでは最も単純な候補として \(a=0\) を考えると,
\[
x^2-4ax+4a+8 \xrightarrow{\,a=0\,} x^2+8
\]
となる。任意の実数 \(x\) に対して \(x^2\ge0\) であるから \(x^2+8>0\) が成り立ち,特に \(2\le x\le 6\) においても常に正である。

\subsubsection*{部分的な結論\\[-15pt]}
以上より,$a=0$ \text{ のとき } $2\le x\le 6$ \text{ において } $x^2-4ax+4a+8>0$が成立する。したがって,$a=0$は問題の答えの領域に含まれる。(この結果は検算に使うとよい。)


% \section*{3}

% (ここに内容を記述)

\end{document}
特別な場合に知ってる例に帰着させられないか

数学的な定義に立ち返る

実際の現象とリンクさせる

別解を考える

感覚的に理解できるような具体的な数値を入れて計算を回す(抽象的な時と同じ流れで)

可視化する

必要条件かだけをチェック

十分条件かだけをチェック